\documentclass[11pt,icelandic]{article}

\usepackage[a4paper]{geometry}
\usepackage[utf8]{inputenc}
\usepackage{babel}
\usepackage{t1enc}
\usepackage{multicol}
\usepackage{amsmath}
\usepackage{amssymb}
\usepackage{enumerate}
\usepackage{hyperref}
%\usepackage{pictex}
\selectlanguage{icelandic}
\columnsep=0.5truecm
\columnseprule=.01truecm
\hoffset=-1.2truecm
\voffset=-2.5truecm
\textwidth=16.5truecm 
\textheight=27.3truecm
\renewcommand{\Bbb}{\mathbb}

\newcommand{\p}{{\partial}}

\newcommand{\C}{{\Bbb C}}
\newcommand{\R}{{\Bbb R}}
\newcommand{\D}{{\Bbb D}}
\newcommand{\M}{{\cal M}}
\newcommand{\I}{{\cal I}}
\newcommand{\F}{{\cal F}}
\renewcommand{\L}{{\cal L}}
\newcommand{\scalar}[2]{\langle#1,#2\rangle}
\newcommand{\norm}[1]{\|#1\|_{\varphi}}
\newcommand{\set}[1]{\{#1\}}
\newcommand{\adjoint}{\bar {\partial}^*_{\varphi}}
\renewcommand{\Re}{\operatorname{Re}}
\renewcommand{\Im}{\operatorname{Im}}
\newcommand{\Li}{\operatorname{Li}}
\newcommand{\Log}{\operatorname{Log}}
\newcommand{\stod}{\operatorname{supp}}
\newcommand{\astand}{{\operatorname{\text{astand}}}}
\newcommand{\nin}{\mbox{$\;\not\in\;$}}
\newcommand{\dive}{\mbox{${\rm\bf div\,}$}}
\newcommand{\curl}{\mbox{${\rm\bf curl\,}$}}
\newcommand{\grad}{\mbox{${\rm\bf grad\,}$}}
\newcommand{\spann}{\mbox{${\rm Span}$}}
\newcommand{\tr}{\mbox{${\rm tr}$}}
\newcommand{\rank}{\mbox{${\rm rank}$}}
\newcommand{\image}{\mbox{${\rm image}$}}
\newcommand{\nullity}{\mbox{${\rm null}$}}
\newcommand{\proj}{\mbox{${\rm proj}$}}
\newcommand{\id}{\mbox{${\rm id}$}}
\newcommand{\Rn}{\mbox{${\bf R}^n$}}
\newcommand{\Rm}{\mbox{${\bf R}^m$}}
\newcommand{\Rk}{\mbox{${\bf R}^k$}}
\newcommand{\Av}{\mbox{${\bf A}$}}
\newcommand{\av}{\mbox{${\bf a}$}}
\newcommand{\uv}{\mbox{${\bf u}$}}
\newcommand{\vv}{\mbox{${\bf v}$}}
\newcommand{\wv}{\mbox{${\bf w}$}}
\newcommand{\xv}{\mbox{${\bf x}$}}
\newcommand{\zv}{\mbox{${\bf z}$}}
\newcommand{\yv}{\mbox{${\bf y}$}}
\newcommand{\bv}{\mbox{${\bf b}$}}
\newcommand{\cv}{\mbox{${\bf c}$}}
\newcommand{\dv}{\mbox{${\bf d}$}}
\newcommand{\ev}{\mbox{${\bf e}$}}
\newcommand{\fv}{\mbox{${\bf f}$}}
\newcommand{\gv}{\mbox{${\bf g}$}}
\newcommand{\hv}{\mbox{${\bf h}$}}
\newcommand{\iv}{\mbox{${\bf i}$}}
\newcommand{\jv}{\mbox{${\bf j}$}}
\newcommand{\kv}{\mbox{${\bf k}$}}
\newcommand{\pv}{\mbox{${\bf p}$}}
\newcommand{\nv}{\mbox{${\bf n}$}}
\newcommand{\qv}{\mbox{${\bf q}$}}
\newcommand{\rv}{\mbox{${\bf r}$}}
\newcommand{\sv}{\mbox{${\bf s}$}}
\newcommand{\tv}{\mbox{${\bf t}$}}
\newcommand{\ov}{\mbox{${\bf 0}$}}
\newcommand{\Fv}{\mbox{${\bf F}$}}
\newcommand{\Gv}{\mbox{${\bf G}$}}
\newcommand{\Uv}{\mbox{${\bf U}$}}
\newcommand{\Nv}{\mbox{${\bf N}$}}
\newcommand{\Hv}{\mbox{${\bf H}$}}
\newcommand{\Ev}{\mbox{${\bf E}$}}
\newcommand{\Sv}{\mbox{${\bf S}$}}
\newcommand{\Tv}{\mbox{${\bf T}$}}
\newcommand{\Bv}{\mbox{${\bf B}$}}
\newcommand{\Oa}{\mbox{$(0,0)$}}
\newcommand{\Ob}{\mbox{$(0,0,0)$}}
\newcommand{\Onv}{\mbox{$[0,0,\ldots,0]$}}
\newcommand{\an}{\mbox{$(a_1,a_2, \ldots,a_n)$}}
\newcommand{\xn}{\mbox{$(x_1,x_2, \ldots,x_n)$}}
\newcommand{\xnv}{\mbox{$[x_1,x_2, \ldots,x_n]$}}
\newcommand{\vnv}{\mbox{$[v_1,v_2, \ldots,v_n]$}}
\newcommand{\wnv}{\mbox{$[w_1,w_2, \ldots,w_n]$}}
\newcommand{\tvint}{\int\!\!\!\int}
\newcommand{\thrint}{\int\!\!\!\int\!\!\!\int}
\renewcommand{\ast}{{\operatorname{\text{astand}}}}

%
\parindent 0pt
\pagestyle{empty}




\begin{document}
%{\large\bf Háskóli Íslands}  \hfill {\large \bf Verkfræði- og náttúruvísindasvið}

%\bigskip
\begin{center}
	{\Large\bf STÆRÐFRÆÐIGREINING IIIA - FORMÚLUBLAÐ}\\
    Benedikt Steinar Magnússon <bsm@hi.is>, \url{https://github.com/benediktmag/formulublad}
\end{center}

%\bigskip
\hrule

\section*{1. Linear Ordinary Differential Equations}

\subsection*{Fird Order Linear Equations}
$y'+p(x)y=g(x)$:
\begin{align*}
 y(x) &= e^{-M(x)}\left(C+\int g(x) e^{M(x)}\, dx\right)
 & \text{where } M(x) = \int p(x) dx.
\end{align*}

\subsection*{Reduction of order}
$y'' + p(x)y' + q(x)y=0$:
\begin{align*}
 u_2(x)&= u_1(x) \int \frac{e^{-P(t)}}{u_1^2(t)}\, dt, 
&\text{where } P(x) =  \int p(x)\, dx.
\end{align*}

\section*{2. Separation of Variables}

\section*{3. Series Solutions of Linear Equations}

\subsection*{Recursive formula for the coefficients at an ordinary point}
\begin{align*}
    (k+2)(k+1)c_{k+2} + \sum_{j=0}^{k} (k-j+1)a_jc_{k-j+1} + \sum_{j=0}^{k}b_jc_{k-j} &= 0. 
\end{align*}

\subsection*{Recursive formula for the coefficients at a regular singular point}
\begin{align*}
P(k+\gamma)c_k + \sum_{j=0}^{k-1} ((j+\gamma)a_{k-j} + b_{k-j}) c_j &=0, &k\in \mathbb N,
\end{align*}
with the understanding that for $k=0$ the above sum is empty, and the indicial 
polynomial $P$ is given by 
\begin{align*}
 P(X) &= X(X-1) + a_0X + b_0.
\end{align*}

\section*{4. Existence Theory}

\subsection*{Picard iteration}
\begin{align*}
  \phi_{m+1}(x) = y_0 + \int_{x_0}^x f(t,\phi_m(t))\, dt.
\end{align*}

\section*{5. The Exponential of a Matrix}

\section*{7. Sturm-Liouville Theory}

\subsection*{Sturm-Liouville form}
\begin{align*}
 Ly = \frac{d}{dx} \left(p(x)\frac{dy}{dx}\right) + q(x)y = g(x)
\end{align*}

\subsection*{Inner product and norm}
\begin{align*}
 \langle u,v\rangle &= \int_a^b u(x)\overline {v(x)}\, dx 
 &  \|u\|_2 = \sqrt{\langle u,u\rangle }
\end{align*}

\section*{-. Fourier series}

 \subsection*{Fourier series $f$ on $[-L,L]$: Exponential form:}
 
 \begin{align*}
  f(x) &\sim \frac{c_0}2 + \sum_{n=-\infty}^\infty c_n e^{\frac{inx\pi}{L},}\\
  c_n &= \hat f(n) = \frac 1{2L} \int_{-L}^L f(x) e^{-\frac{n\pi x}L}\, dx, & n\in \mathbb Z.
 \end{align*}


 \subsection*{Fourier series $f$ on $[-L,L]$: Trigonometric form}

 \begin{align*}
  f(x) &\sim \frac{a_0}2 + \sum_{n=1}^\infty a_n \cos\left(\frac {n\pi x}L \right) + \sum_{n=1}^{\infty}b_n \sin\left(\frac{n\pi x}L \right)\\
  a_n &= \frac 1L \int_{-L}^{L} f(x) \cos\left(\frac{n\pi x}L\right)\, dx, & n\geq 0\\
  b_n &= \frac 1L \int_{-L}^L f(x) \sin\left(\frac{n\pi x}L \right)\, dx, & n\geq 1
 \end{align*}

 \subsection*{Solving Ordinary Differential Equations}
  If $P(D)u=f$, then $P(in)\hat u(n) = \hat f(n)$ and
\[ u(x) = \sum_{n=-\infty}^{\infty} \frac{\hat f(n)}{P(in)}e^{inx}.\]
 
\end{document}
